\documentclass[a4paper,10pt]{article}

%\usepackage[latin1]{inputenc}
\usepackage[utf8]{inputenc}
\usepackage[swedish]{babel}

\usepackage{xspace,graphicx}
\usepackage{url}
\usepackage{fancyhdr}
\usepackage{textcomp}
\usepackage[colorinlistoftodos]{todonotes}

\usepackage{color}
\newcommand{\RED}[1]{\textcolor{red}{#1}\xspace}
\newcommand{\BLUE}[1]{\textcolor{blue}{#1}\xspace}

\frenchspacing
\setlength{\parindent}{0pt}
\setlength{\parskip}{1ex plus .5ex minus .5ex}

\newcommand{\Todo}[2][color=magenta!40]{\todo[#1]{\fontsize{7pt}{6pt}\fontfamily{pag}\selectfont#2}}
\newcommand{\BA}[1]{\Todo[color=blue!20,linecolor=blue!66]{\textbf{B}: #1}}

\title{Manus för presentationen den 4/3 2015}
\author{}
\date{\today}

\begin{document}
\maketitle
%\tableofcontents

\section{Ämnesområdet (5 min.)}

\begin{itemize}
\item Vad är datalogi och vad är programvaruteknik?
\item Datalogi är ett kärnämne och grunden för alla andra ämnen
  som undervisas på DSV. Programmering fungerar dessutom som
  ingång till ``formellt tänkande'' \BA{Fyll gärna på med vad du
    tänkte, Jozef.} och är därför en viktig övning
  även inför modellering etc.
\item Historik. DSV har historiskt sett haft ett starkare fokus på
  datalogi och programvaruteknik. \BA{Fyll gärna på med mer vad du
    tänkte, Jozef.} Ämnet har hamnat ``under radarn'' då vi
  t.ex. inte har någon representant i
  seniorforskarkollegiet. Detta beror till exempel på Terttu Orcis
  bortgång, etc.
\item Gruppen för Datalogi och Programvaruteknik är en del av ACT
  och står för stora delar av den datalogiska och
  programvarutekniska verksamheten på DSV. Framförallt gäller
  detta undervisning men vi sysslar även med forskning inom ämnet.
  Datalogi och programvaruteknik finns dock, eftersom det är
  grunden för all verksamhet på DSV, utspridd på flera
  enheter. Det finns några grupper som förutom vår som arbetar med
  mer renodlat datalogiska frågor.
\end{itemize}

\section{Vad/vem är gruppen för D\&PVT? (10 min.)}

\begin{itemize}
\item Personer. Beatrice Åkerblom, Henrik Bergström, Jozef
  Swiatycki, Peter Idestam-Almquist, Pierre Wijkman, Stefan
  Möller.
\item Forskning. 
  \begin{itemize}
  \item I gruppen har vi en person som bedriver aktiv forskning på
    doktorandnivå. Arbetet görs inom
    programmeringsspråksområdet. \BA{Jag skall förtydliga och
      specificera detta.}
  \end{itemize}
\item Undervisning.
  \begin{itemize}
  \item Gruppen har ett stort intresse för pedagogik och försöker
    ständigt hitta nya sätt att undervisa som passar datalogiämnet
    och programvarutekniken. 
  \end{itemize}
\end{itemize}

\section{D\&PVT och rollen på DSV (10 min.)}

\begin{itemize}
\item Vi ger obligatoriska (?) kurser på alla program \BA{Sant,
    Jozef?} och många andra verksamheter (kurser och forskning) är
  direkt beroende av det arbete vi gör.
\item Ett av våra program är datavetenskapligt.
  \begin{itemize}
  \item DVK-programmet nu.
  \item DVK-programmet i ``samhället''.
  \item DVK-programmet i framtiden.
  \end{itemize}
\item Vi handleder uppsatser för studenter med datalogiskt eller
  programvarutekniskt intresse.
\end{itemize}

\section{Problem?}

\begin{itemize}
\item Ingen professor $\rightarrow$ inga doktorander $\rightarrow$
  inga handledare.
\item 
\end{itemize}

\end{document} 
